\documentclass[12pt, a4paper]{article}
\usepackage{graphicx}
\graphicspath{ {images/} }
\usepackage{amsmath} % many things
\usepackage{physics} % (partial) derivatives, etc.
\usepackage{siunitx}
\usepackage{hyperref}
\hypersetup{
    colorlinks=true,
    linkcolor=black,
    filecolor=black,
    urlcolor=blue,
    citecolor=black,
}
\usepackage{amssymb}
\usepackage{subfigure}



\let\oldexp\exp
\renewcommand{\exp}[1]{\ensuremath{\oldexp \left( #1 \right)}}
%\renewcommand{\exp}[1]{\ensuremath{\left(#1\right)}}

\textwidth=170mm
\textheight=250mm
\hoffset= -20mm       % may need change
\voffset= -25mm       % may need change

\begin{document}
%% we create our own title page
\thispagestyle{empty}     % only for frontpage
\null\vspace{40mm}
\begin{center}
{%%%%%%%%%%%%%%%%%%%%%%%%%% Title
\Large  Magneto-Optical Trap
\footnote{\noindent Experiment F20, performed on 26\textsuperscript{th} August 2019,
Supervisor Saba Zia Hassan,
short special evaluation}
}\\[15mm]
%%%%%%%%%%%%%%%%%%%%%%%%%%% Authors
L. Hahn and L. Kuehmichel

\vspace{25mm}

\parbox{0.9\textwidth}{
Abstract:    
\small The abstract should preferentially be in English. Here we explain in a
few lines (i) what was done, and (ii) what the results were.
}
\end{center}

\vfill
Audited as a special evaluation: Date, Signature:
\vspace{20mm}

%% Empty backside of title page, remove for single-sided printing
% \newpage  
\null\thispagestyle{empty} 
   
%\newpage
%\tableofcontents 

\newpage

\pagenumbering{arabic} %% start page 1 
\section{Introduction: Physical Subjects and Goal of the Experiment}
\subsection{Basic Overview of Physical Subjects}
The Magneto-Opitcal Trap makes use of an abundance of physical topics mainly focussed around atomic pyhsics, optics and spectroscopy. The following section aims to provide an overview of these topics without an in-depth explanation. The source of all the information summarised in Chapter 1 is \cite{script}, where one can find more in-depth information on the subject.
\subsubsection{Level Structure of Rubidium Atoms and Laser Spectroscopy}
\begin{figure}
    \centering
    \parbox{0.4\textwidth}{
        \centering
        \includegraphics[width=0.4\textwidth]{800px-Doppler_spectroscopy.png}
    }

    \caption{Schematic sketch of the spectroscopy path with:  PBS being a polarizing beam splitter, GT means Galilean Telescope and H references a half waveplate, \cite{script}}
    \label{doppler_spect}
\end{figure}

As shown in Fig.\ref{doppler_spect}, the laser beam passes a Rb vapour cell before hitting a photo diode, where the intensity is evaluated. Simultaneously, the laser scans through various frequencies to measure an intensity response. As the laser propagates through the gas, the stimulated atomic state transitions have an effect on the laser intensity following the Lambert-Beer law:
\begin{equation}
\frac{dI}{dx}= -\kappa I
\end{equation}
where $\kappa = h\nu n_0 \alpha (P_0 - P_1)$ denotes the absorption coefficient, $x$ the path and $I$ the beam intensity. Furthermore:
\begin{equation}
\alpha = \alpha_0 \mathcal{L}(\nu , \nu_0)
\end{equation}
describes the frequency dependence of $\kappa$, which has a Lorentzian shape. The expression $P_0 - P_1$ gives us the relation between atoms in the excited ($P_1$) versus the ground state ($P_0$). The fraction $\frac{P_1}{P_0}$ has a Gaussian shape and depends on frequency and temperature.
A complication for evaluating the exact position of absorption dips are the Doppler shifts created through the movement of the atoms in the vapour cell. The so-called "Doppler broadening" which we then can observe in our Intensity-Frequency-diagram has a Gaussian shape with the standard deviation being:
\begin{equation}
\sigma_v = \sqrt{\frac{k_B T}{m}}
\end{equation}
where $T$ denotes the temperature, $k_B$ the Boltzmann constant and $m$ the mass of a particle. Through all these considerations, we can then describe the different populations of $P_0$ and $P_1$ accordingly and arrive at the Doppler free spectroscopy: instead of using one single beam, we use a probing beam and a pumping beam which are detuned in the order of a few Megahertz to evaluate the Doppler shift dependeny. The interplay of these two beams creates a so-called "saturated absorption dip" right at the resonance frequency without doppler shifting but depends on $\frac{I}{I_{sat}}$ with $I_{sat}$ being the saturation intensity, which depends on spontaneous emission rate of the atoms and $\alpha$.

\begin{figure}
    \centering
    \parbox{0.4\textwidth}{
        \centering
        \includegraphics[width=0.4\textwidth]{Rb_satabs.png}
    }

    \caption{measured spectra with cross-over resonances, \cite{script}}
    \label{measured_spec}
\end{figure}

Since the Rubidium atoms have a more complicated V-shaped energy level schematic (two upper energy levels and one lower), the process creates crossover-peaks which can be observed in the Gaussian absorption dip.
Rubidium has fine structure levels with L-S coupling
\begin{equation}
\vec{J} = \vec{L} + \vec{S}
\end{equation}
($\vec{J}$ = total electronic angular momentum, $\vec{L}$ = total orbital angular momentum, $\vec{S}$ = total electronic spin angular momentum)

and hyperfine structure levels with I-J coupling
\begin{equation}
\vec{F} = \vec{I} + \vec{J}
\end{equation}
($\vec{F}$ = total angular momentum, $\vec{I}$ = spin angular momentum)
which complicate the doppler-free absorption spectrum even further.
All these effects can be seen in accumulation in Fig.\ref{measured_spec}

Also, we make use of frequency modulation spectroscopy in order to measure the Lamb-dips and cross-overs as well as to stabilise the laser in the MOT-part of the experiment. Through a series of effects, one gets a derivative of the absorption signal and therefore can estimate the position of the absorption dip more accurately. This whole process is done in the spectroscopy-branch of the experimental setup. \cite{script}

\subsubsection{Magneto-Optical Trap}
The Magneto-Optical trap used to cool atoms to a few $100\ \mu  K$ uses two main principles: the effect of optical molasses and the Zeeman-effect, which occurs under the influence of an external magetic field gradient. The linear momentum imposed upon atoms by light can be written as $p=\hbar k$. Furthermore $\hbar \omega$ changes the internal energy of an atom and the angular momentum has effect on the orbital angular momentum $l$ with the selection rule $\Delta l=\pm 1$. Through these elementary processes, the atoms are slowed down in the optical molasses (OM): two opposing standing-wave laser beams dampen the atomic motion where the force $\vec{F}_{OM}$ is proportional to the velocity of the atoms $-\vec{v}$. This process only works in cooperation with the doppler-free spectroscopy, since the resonance frequencies must be "scanned" and acquired.

However, there are limits to cooling via OM: the recoil limit (for $^{85}Rb$, $T_r = 0.370\mu K$) and doppler temperature limit (for $^{85}Rb$, $T_D = 143.41\mu K$), which limit the cooling to a few hundreds of $\mu K$.

The MOT is comosed of a quadrupole magnetic field and three pairs of counter propagating laser beams.
The actual trapping is achieved through a anti-Helmholtz-coil, which makes use of the $M_e$-Zeeman-component-splitting to $M_e = +1$ and $M_e = -1$ depending on the polarisation of the light and to which direction the atoms are escaping.

The missing link between measuring the temperature of our MOT and the physics mentioned above is the so called "release-recapture method" which has as its base the rate equation:
\begin{equation}
\frac{dN}{dt} = L - \alpha N
\end{equation}
where $N$ denotes the number of trapped particles, $L$ the loss-rate and $\alpha$ the one-body loss coefficient. The solution is given by:
\begin{equation}
N(t) = \frac{L}{\alpha}(1-e^{-\alpha x})
\end{equation}
Here, the laser is switched on and off to analize the number of recaptured atoms after a varying time interval, in which the laser is turned off.
Through the Stefan-Boltzmann velocity distribution, one arrives at the follwing fraction in order to compute the temperature $T$:
\begin{equation}
\frac{N(t)}{N(0)} = erf(\chi) - \frac{2}{\sqrt{\pi}} \chi e^{-\chi ^2}
\end{equation}
where $\chi = \sqrt{\frac{M}{k_{b} T}}\frac{R}{t}$, $t$ is the time stamp at which the MOT is turned back on and $N(0)$ is the initial atom number (where the trap is fully loaded). \cite{script}


\subsection{Goal of the Experiment}
The goal of the experiment is to analise doppler-free spectra and match atomic transitions to dips in our measurements as well as to calculate the MOT-temperature with the trap turned off. In the second part of the experiment we acquire $\alpha$ and $L$ in dependency of different detuning frequencies and Helmholtz-coil currents in order to generate the parameters which maximise the number of trapped atoms. Then, through a release-recapture measurement, we calculate the temperature of the MOT.


\section{Experimental Setup}

\begin{figure}[h]
\centering
    \subfigure[MOT setup]{\includegraphics[width=0.4\textwidth]{779px-Vacuum_chamber_2.png}}
    \subfigure[optical setup]{\includegraphics[width=0.4\textwidth]{800px-Setup-schematic.png}}
\caption{Schematics of the experimental setup, \cite{script}}
	\label{expsetup}
\end{figure}









%%% LARS ANFANG %%%
\newpage

\section{Data Analysis}
\subsection{Spectroscopy}
\subsubsection{Multiplet Separation}

During the experiment, all data was recorded as a voltage rather than frequency, energy, et cetera. Therefore, to analyse the spectral data of the Rubidium 85 and 87 D2 lines, we first need to calibrate the horizontal axis. In all atoms, energy levels are only observable through absorption or emission of a photon by an electron jumping from one level to another. Due to this, it is not possible to know the absolute energy a level possesses. Only the difference between two levels is observable. As such, we can only infer a cardinal interval scale to the horizontal axis, meaning the zero point is arbitrary.

First, we plotted the full spectrum in which all fine energy level dips are visible. The plot looked as we anticipated, showing 4 separated large (Gaussian) dips, each accommodated with several smaller (Lorentzian) peaks near their middle. In order to find the mean value each dip is at, we fitted a Gaussian added with a linear background function to each of the profiles:

\begin{equation}
A_0 \exp{\frac{-(x - \mu)^2}{2 \sigma^2}} + ax + b
\label{gaussianfit}
\end{equation}

Where $A_0$, $\mu$, $\sigma$, $a$, and $b$ were fit parameters, with $A_0$ representing the (negative) amplitude, $\mu$ the dip's mean value, $\sigma$ the standard deviation and $a$ and $b$ being background parameters. In order to achieve a better fit, the areas with the Lorentzian peaks were masked out beforehand.

\begin{figure}
    \centering
    \parbox{0.45\textwidth}{
        %\centering
        \includegraphics[width=0.5\textwidth]{fullspectrum.png}
    }
    \hfill
    \parbox{0.45\textwidth}{
        %\centering
        \includegraphics[width=0.5\textwidth]{fullspectrumgaussian}    
    }
    \caption{Left: Full Spectrum. Right: Gaussian Fits}
\end{figure}

Using the separations in $(\mu \pm \sigma)$ from one another, we can derive the calibration from the literature value of these separations. With the literature value\footnote{$= 6.8346826109042(90)\;\si{\giga\hertz}$\cite{script}} for the larger separation between Rubidium 87 $F = 1$ and $F = 2$, we can calculate

\begin{equation}
\alpha^{-1} \equiv \dv{V}{f} = (1.9366 \pm 0.0008) \cdot 10^{-10} \quad \si{\frac{\volt}{\hertz}}
\end{equation}

We then assume a linear proportionality $f \propto \alpha V$ and use $\alpha$ to convert any difference in volts $\Delta V$ to a difference in Hertz $\Delta f$:

\begin{equation}
\Delta f (\Delta V) = \alpha \Delta V
\label{conversion}
\end{equation}

Comparing the literature value of the separation in the Rubidium 85 $F = 2$ and $F = 3$ dips\footnote{$=3.0357324390(60) \;\si{\giga\hertz}$\cite{script}} with the one calculated from \autoref{conversion}, we can estimate an uncertainty in $\alpha$, which we found to be $(0.71 \pm 0.06) \%$. This is sufficiently accurate for the purposes of this experiment.

\subsubsection{Natural Transition Line Width and Doppler Broadening}

We then analysed the spectra we recorded for each individual Gaussian dip. Once again using \autoref{gaussianfit} as the fit function while masking out the Lorentzian peaks, we received a Gaussian profile which we could then subtract from the data in order to further analyse the Lorentzian peaks.

\begin{figure}
    \centering
    \parbox{0.45\textwidth}{
        \includegraphics[width=0.5\textwidth]{gaussian1.png}
    }
    \hfill
    \parbox{0.45\textwidth}{
        \includegraphics[width=0.5\textwidth]{gaussian2.png}
    }
    \parbox{0.45\textwidth}{
        \includegraphics[width=0.5\textwidth]{gaussian3.png}
    }
    \hfill
    \parbox{0.45\textwidth}{
        \includegraphics[width=0.5\textwidth]{gaussian4.png}
    }
    \caption{Gaussian Profiles}
    \label{gaussianprofiles}
\end{figure}

From the Standard Deviation of these Gaussian Profiles we can also calculate the temperature of the atomic sample:

\begin{equation}
T = \sigma^2 \frac{mc^2}{\nu_0^2 k_B}
\end{equation}

Where $m$ is the atomic mass of a single Rubidium atom, $c$ is the speed of light, $\nu_0 = \frac{c}{\lambda}$ is the laser frequency and $k_B$ is the Boltzmann constant. Since we are not yet cooling the sample, we would expect values around $300\si{\kelvin}$. The individual results are shown in \autoref{gaussianprofiles}.










\newpage

%%% LARS ENDE %%%




















%% Bilder: Sie sollten nicht versuchen, postscript 'from scratch' selbst zu
%% schreiben, sondern den Output von Analyseroutinen (PAW, ROOT, ...) oder
%% von Plot-Programmen (xfig, tgif) verwenden

\begin{figure}[h]
  \centering
  %\includegraphics[width=8cm,bbllx=112,bburx=447,bblly=264,bbury=582]{fig4.ps}
  %\includegraphics[width=8cm]{fig4.eps}  
  \caption{Beispiel f\"ur eingebundenes Postscript-File}
  \label{ps}
\end{figure}

Als erstes machen wir mal eine Gleichung im fortlaufenden Text (sog.\ math mode in LATEX): $s=\frac{1}{2}
a t^2$, dann kommt eine separate gesetzte (sog.\ displaymath in LATEX):
\[         s=\frac{1}{2}a t^2              \]  % die Leerzeichen tun nichts
und dann geht's weiter im Text. Vielleicht eine Gleichung mit fortlaufender 
Nummer:
\begin{equation}
 s(t_1)=s(0) + \int_0^{t_1} v(t)\mbox{d}t 
\end{equation}
Symbole in Formeln werden in einem besonderen, kursiven Font
gesetzt, au\ss er Funktionen (z.B. $\sin\Theta$) und Diffentialzeichen 'd' 
(deswegen die Klimmz\"uge in obiger Formel im LATEX). Wenn man die Symbole im
Text benutzt, z.B. den Weg $s$, dann macht man das auch mit {\em math mode}.
Hervorhebungen in {\em kursiv}, oder {\bf fett}. Es gibt nat\"urlich auch 
andere
Schrifttypen z.B.\ {\sf Sans Serif}.

Hier noch etwas zu Bindestrichen: das - ist ein Binde-Strich f\"ur 
zusammengesetzte Haupt\-w"or\-ter, das -- oder --- ein Gedankenstrich. 

Optionale
Trennungen am Zeilen$\backslash$-ende im LATEX so (siehe z.B.\ das vorige 
`Haupt\-w"or\-ter') im LATEX, falls es zu seltsamen
Trennungen oder Nichttrennungen (oft bei deutschen Umlauten) kommt.

















%%% LARS ANFANG %%%

\newpage

\section{Discussion and Critical Analysis}

% regarding the temperature deviations from the gaussian profiles
It is noteworthy that while these are within the correct order of magnitude, their deviation from the expected $300\si{\kelvin}$ is quite large, suggesting that there is an underlying systematical uncertainty. Considering that the wider the range of the Lorentzian dips we had to mask is, the further the results stray from the expected value, we can assume that one of the most influential sources of this uncertainty is an inaccuracy in the fit. This is further confirmed by the fact that any uncertainty in $T$ stems purely from an uncertainty in $\sigma$ (with quadratic dependence), which is a fit parameter.



\newpage
%%% LARS ENDE

































Hier werden alle wesentlichen Ergebnisse nochmals angefuehrt und diskutiert.
Bild~\ref{ps} 
% \ref{ps} korrespondiert mit \label{ps}. Das kann man fuer
% Bilder, Tabellen oder Abschnitte [\section, \subsection ...] benutzen und
% muss nicht bei jeder Umstellung neu zaehlen und editieren.
ist ein Beispiel fuer ein eingebundenes .ps (nicht .eps).
Hier muss zus\"atzlich eine sog. {\em bounding box} eingegeben werden, das 
sind die Bildgrenzen in Pixels, hier\\
\centerline{\tt bbllx=112,bburx=447,bblly=264,bbury=582[,clip=1]}
(bounding box lower left x, upper right x, dito y). Die {\it bounding box} 
k\"onnen Sie beim Postscriptviewer gv mit dem Cursor ablesen. {\tt clip=1'}
schneidet Grafik au\ss erhalb der {\it bounding box} ab, andernfalls 
k\"onnen werden auch Bildteile au\ss erhalb wiedergegeben (und \"uberschreiben
ggf. Text etc.).

Am Schluss kann man noch eine allgemeinere Bemerkung zum Versuch machen.


\newpage 
%% hier wird 'von Hand' eine neue Seite erzwungen

%% Literatur)

\begin{thebibliography}{00}   % {00}: max 2-stellige Referenznummer
\bibitem{script} C. Hofmann, M. Repp, V. Gavryusev , S. H\"afner, \href{https://www.physi.uni-heidelberg.de/Forschung/QD/f20wikinew/index.php/Main_Page}{F20-Wiki}, Fortgeschrittenenpraktikum Heidelberg, Faculty for Physics, University of Heidelberg
\bibitem{afo} F. Afo, Nature 15 (1905) 23
\bibitem{uwe} Uwe Ludwig, private Mitteilung
\bibitem{karl} Karl Popper, Phys.~Rev.~Lett.~95 (2001) 25
\bibitem{dipl} K. Winter, Diplomarbeit Heidelberg (1968)
\bibitem{bibel} Genesis 3,4

\end{thebibliography}

\end{document}